\documentclass{amsart}


\begin{document}
\title{Algoritmi e Strutture Dati\\
Statistica per i big data\\
Esame del 3 Settembre 2020
}


\newcommand{\NomeStudente}{Giuseppe Persiano}
\newcommand{\nomeClasse}{solb}


\maketitle

\hfill{{\bf Studente: \NomeStudente}}

\smallskip
La classe BST che abbiamo discusso in classe fornisce una semplice 
implementazione della struttura dati albero binario di ricerca.
In questo problema d'esame si richiede di estendere la classe {\sf{BST}}
in una nuova classe, che chiamiamo {\tt{\nomeClasse}}, che fornisce un metodo 
{\tt somma}. Il metodo {\tt somma} restituisce la somma di tutti
valori presenti nell’albero.


\medskip\noindent {\bf Istruzione per la consegna.}
Tutto il codice consegnato deve essere contenuto nel file
{\tt \nomeClasse .py} ed inviato per e-mail all'indirizzo
{\tt giuper@gmail.com} prima delle ore 11 di oggi, 3 Settembre 2020.
Il file deve contenere una classe {\tt \nomeClasse}.

Il file {\tt driver.py} pu\`o essere usato per controllare la correttezza
dell'elaborato. 
La commissione pu\`o ovviamente verificare la correttezza dell'elaborato
usando anche altre istanze del problema.

\end{document}
