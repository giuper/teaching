\documentclass{amsart}


\usepackage{fullpage}
\usepackage{graphicx}



\begin{document}
\title{Algoritmi e Strutture Dati\\
Statistica per i big data\\
Esame del 4 Gennaio 2021
}


\newcommand{\NomeStudente}{Giuseppe Persiano}
\newcommand{\nomeClasse}{{\tt{Sol}}}
\newcommand{\nomeMetodo}{{\tt \_\_eq\_\_}}
\newcommand{\oraconsegna}{10:35}
\newcommand{\dataoggi}{4 Gennaio, 2021}


\maketitle

\hfill{{\bf Studente: \NomeStudente}}

\smallskip
L'ufficio statistico del comune di Bugliano, sebbene il
comune conti solo 1323 abitanti,
ha deciso di voler usare la classe {\tt bignum} in modo da 
poter trattare interi arbitrariamente grandi.
Il direttore dell'ufficio si rende conto che la classe non \`e completa
e vi chiede di costruire una nuova classe
{\tt eqbigNum} che, oltre ai metodi della classe 
{\tt bigNum}, implementa l'operatore \nomeMetodo\
che controlla se due {\tt bignum} sono uguali.

\medskip\noindent{\bf Materiale della traccia.}
La cartella contiene il pdf di questa traccia, i file
{\tt linkedList.py} e {\tt bigNum.py} che contengono le classi sviluppate
in classe, il file {\tt driver.py} che pu\`o essere usato
per verificare il funzionamento della classe progettata,
e il file  {\tt result.txt} che contiene l'output atteso di {\tt driver.py}.

\medskip\noindent{\bf Istruzione per la consegna.}
Tutto il codice consegnato deve essere contenuto nel file
{\tt Sol.py} ed inviato per e-mail all'indirizzo
{\tt giuper@gmail.com} prima delle ore \oraconsegna\ di oggi, 
\dataoggi. Non inviare altri file e n\'e tantomeno file zip.
Il file consegnato deve poter essere usato per eseguire
il codice di {\tt driver.py}. 
Si pu\`o assumere che i file {\tt linkedList.py} e {\tt bigNum.py} 
siano presenti al momento dell'esecuzione.


\end{document}
