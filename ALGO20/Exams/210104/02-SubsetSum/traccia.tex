\documentclass{amsart}

\usepackage{fullpage}


\begin{document}
\title{Algoritmi e Strutture Dati\\
Statistica per i big data\\
Esame del 4 Gennaio 2021
}


\newcommand{\NomeStudente}{Giuseppe Persiano}
\newcommand{\nomeClasse}{{\tt{Sol}}}
\newcommand{\nomeMetodo}{{\tt{Solve}}}
\newcommand{\oraconsegna}{10:35}
\newcommand{\dataoggi}{4 Gennaio, 2021}


\maketitle

\hfill{{\bf Studente: \NomeStudente}}

\smallskip
Vittorino, il capo dei vigili urbani del comune di Bugliano, passa il tempo,
tra una multa e l'altra, a studiare il codice {\tt python} che trova
sul sito del corso. Ultimamente ha trovato interessante il problema 
del Subset Sum ma ritiene uno spreco il poter utilizzare ogni elemento
della lista input $L$ una sola volta.
La sua coscienza ecologista gli impone di dover riutilizzare gli elementi
e quindi vi commissiona l'implementazione di una classe python \nomeClasse.
Gli oggetti di questa classe, oltre alla lista $L$ ed al target $t$,
hanno come attributo una lista $M$ di molteplicit\`a:
il valore $M[i]$ indica il numero massimo di volte che $L[i]$ pu\`o essere 
usato per raggiungere il target $t$.
Questa classe deve offrire, oltre al costruttore che prende
in input $L,M$ e $t$, 
il metodo \nomeMetodo\ che applicato ad un oggetto ci dice se esiste
o no soluzione.


Ad esempio
\begin{itemize}
\item Consideriamo $L=[2,3]$ ed $M=[1,2]$ e quindi
        abbiamo due elementi $2$ e $3$ ma il $3$ pu\`o essere riutilizzato.

\noindent
    Possiamo ottenere il target $t=8$ usando una volta il $2$
        e due volte il $3$. Infatti $2+3+3=8$.

\noindent
    Anche il target $t=5$ pu\`o essere raggiunto sommando $2+3$.

\noindent
    Invece il target $t=4$ non pu\`o essere raggiunto. Nota che $2+2=4$ non
    \`e una soluzione in quanto $L[0]=2$ pu\`o essere usato una sola volta
    come indicato da $M[0]=1$.

\end{itemize}

\medskip\noindent{\bf Materiale della traccia.}
La cartella contiene il pdf di questa traccia, i file
{\tt stack.py, back.py, subsetSum0.py},
il file {\tt driver.py} che pu\`o essere usato
per verificare il funzionamento della classe progettata
e il file  {\tt result.txt} che contiene l'output atteso di {\tt driver.py}.

\medskip\noindent{\bf Istruzione per la consegna.}
Tutto il codice consegnato deve essere contenuto nel file
{\tt \nomeClasse .py} ed inviato per e-mail all'indirizzo
{\tt giuper@gmail.com} prima delle ore \oraconsegna\ di oggi, 
\dataoggi. Non inviare altri file e n\'e tantomeno file zip.
Il file deve contenere la classe {\tt \nomeClasse}  
che pu\`o essere usata per eseguire
il codice di {\tt driver.py}. Si pu\`o assumere che i file
{\tt stack.py, back.py, subsetSum0.py},
siano presenti al momento dell'esecuzione. 

\end{document}
