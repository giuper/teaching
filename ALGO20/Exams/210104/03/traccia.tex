\documentclass{amsart}


\usepackage{fullpage}


\begin{document}
\title{Algoritmi e Strutture Dati\\
Statistica per i big data\\
Esame del 4 Gennaio 2021
}


\newcommand{\NomeStudente}{Giuseppe Persiano}
\newcommand{\nomeClasse}{{\tt{Sol}}}
\newcommand{\nomeMetodo}{{\tt{contaDisp}}}
\newcommand{\oraconsegna}{10:35}
\newcommand{\dataoggi}{4 Gennaio, 2021}


\maketitle

\hfill{{\bf Studente: \NomeStudente}}

\smallskip
Il nostro amico Capparaccio ha letto dei numeri di Fibonacci ed ha avuto un'idea
che lo tiene sveglio la notte: ha inventato i numeri $k$-acci.
Per $n>0$ e $k\geq 2$, l'$n$-esimo numero Capparaccio di ordine $k$,
$C_{n,k}$, \`e definito nel modo seguente:

$$C_{n,k}=\begin{cases}
        1, & \text{se } 0<n\leq k;\cr
        C_{n-1,k}+C_{n-2,k}+\ldots,C_{n-k,k}, & \text{se } n>k.\cr
\end{cases}
$$
Quindi, invece di sommare i due interi precedenti della sequenza come
nei numeri di Fibonacci (un suo lontano parente) sommiamo i $k$ interi
precedenti.
Ad esempio, per $k=3,4$ abbiamo
$$
\begin{array}{rcl}
C_{1,3}&=&1\\
C_{2,3}&=&1\\
C_{3,3}&=&1\\
C_{4,3}&=&1+1+1=3\\
C_{5,3}&=&1+1+3=5\\
C_{6,3}&=&1+3+5=9\\
C_{7,3}&=&3+5+9=17\\
C_{8,3}&=&5+9+17=31\\
C_{9,3}&=&9+17+31=57\\
\ldots& &\ldots
\end{array}
\qquad
\qquad
\begin{array}{rcl}
C_{1,4}&=&1\\
C_{2,4}&=&1\\
C_{3,4}&=&1\\
C_{4,4}&=&1\\
C_{5,4}&=&1+1+1+1=4\\
C_{6,4}&=&1+1+1+4=7\\
C_{7,4}&=&1+1+4+7=13\\
C_{8,4}&=&1+4+7+13=25\\
C_{9,4}&=&4+7+13+25=49\\
\ldots& &\ldots
\end{array}
$$

Il nostro amico Capparaccio \`e interessato a studiare la percentuale di 
numeri $k$-acci che sono dispari e vi chiede di sviluppare una 
funzione \nomeMetodo\ che prende in input $N,k$ e restituisce la 
frazione di numeri dei primi $N$ numeri $k$-acci che sono dispari.
Capparaccio vi avverte che intende usare la vostra funzione con $N$ molto
grande e quindi voi capite che non \`e possibile usare un algoritmo
ricorsivo per calcolare i numeri $k$-acci (se lo fate, dovrete aspettare
che il vostro programma termini l'esecuzione prima di poter sostenere 
la prova orale).


\medskip\noindent{\bf Materiale della traccia.}
La cartella contiene il pdf di questa traccia, il file
{\tt fibo.py} che contiene le implementazione dei numeri di Fibonacci
che abbiamo discusso in classe e 
il file {\tt driver.py} che usa la funzione \nomeMetodo\ che avete
progettato, e il file  {\tt result.txt} che contiene
l'output atteso di {\tt driver.py}.

\medskip\noindent{\bf Istruzione per la consegna.}
Tutto il codice consegnato deve essere contenuto nel file
{\tt \nomeClasse .py} ed inviato per e-mail all'indirizzo
{\tt giuper@gmail.com} prima delle ore \oraconsegna\ di oggi, 
\dataoggi. Non inviare altri file e n\'e tantomeno file zip.
Il file deve contenere la funzione
\nomeMetodo\ e che pu\`o essere usata per eseguire
il codice di {\tt driver.py}. 




\end{document}
