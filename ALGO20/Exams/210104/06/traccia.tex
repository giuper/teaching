\documentclass{amsart}

\usepackage{fullpage}


\begin{document}
\title{Algoritmi e Strutture Dati\\
Statistica per i big data\\
Esame del 4 Gennaio 2021
}


\newcommand{\NomeStudente}{Giuseppe Persiano}
\newcommand{\nomeClasse}{{\tt{Sol}}}
\newcommand{\nomeMetodo}{{\tt{Solve}}}
\newcommand{\oraconsegna}{10:35}
\newcommand{\dataoggi}{4 Gennaio, 2021}


\maketitle

\hfill{{\bf Studente: \NomeStudente}}

\smallskip
Vittorino, il capo dei vigili urbani del comune di Bugliano, passa il tempo,
tra una multa e l'altra, a studiare il codice {\tt python} che trova
sul sito del corso. Ultimamente ha trovato interessante il problema 
del Subset Sum ma ha una strana fissazione: 
vuole che due elementi consecutivi di $L$, $L[i]$ e $L[i+1]$,
possano usati per raggiungere il target 
solo se $L[i]<L[i+1]$.

Ad esempio
\begin{itemize}
\item Consideriamo $L=[3,7,6,19,2]$.

\noindent
    Possiamo ottenere il target $t=9$ come $9=3+6$ mentre invece
        non possiamo ottenere $t=13$. Infatti l'unico modo per ottenere
        $13$ sarebbe di sommare $7$ e $6$ che per\`o sono consecutivi 
        e $7>6$.

\noindent
    Osserva anche che \`e possibile ottenere $t=5$ come $5=3+2$ perch\`e
        $3$ e $2$ non sono consecutivi e quindi non esiste alcun vincolo.
\end{itemize}
Il vostro compito \`e di implementare la classe \nomeClasse\ che pu\`o
essere usata per risolvere questa variante di Subset Sum.

\medskip\noindent{\bf Materiale della traccia.}
La cartella contiene il pdf di questa traccia, i file
{\tt stack.py, back.py, subsetSum0.py},
il file {\tt driver.py} che pu\`o essere usato
per verificare il funzionamento della classe progettata
e il file  {\tt result.txt} che contiene l'output atteso di {\tt driver.py}.

\medskip\noindent{\bf Istruzione per la consegna.}
Tutto il codice consegnato deve essere contenuto nel file
{\tt \nomeClasse .py} ed inviato per e-mail all'indirizzo
{\tt giuper@gmail.com} prima delle ore \oraconsegna\ di oggi, 
\dataoggi. Non inviare altri file e n\'e tantomeno file zip.
Il file deve contenere la classe {\tt \nomeClasse}  
che pu\`o essere usata per eseguire
il codice di {\tt driver.py}. Si pu\`o assumere che i file
{\tt stack.py, back.py, subsetSum0.py},
siano presenti al momento dell'esecuzione. 

\end{document}
