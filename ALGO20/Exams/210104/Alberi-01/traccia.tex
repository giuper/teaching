\documentclass{amsart}


\usepackage{fullpage}


\begin{document}
\title{Algoritmi e Strutture Dati\\
Statistica per i big data\\
Esame del 4 Gennaio 2021
}


\newcommand{\NomeStudente}{Giuseppe Persiano}
\newcommand{\nomeClasse}{{\tt{Sol}}}
\newcommand{\nomeMetodo}{{\tt{nuovo}}}
\newcommand{\oraconsegna}{10:35}
\newcommand{\dataoggi}{4 Gennaio, 2021}


\maketitle

\hfill{{\bf Studente: \NomeStudente}}

\smallskip
L'ufficio Big Data del comune di Bugliano utilizza la struttura dati alberi
discussa a lezione.
In questa struttura dati si pu\`o inserire un valore specificando
il percorso dalla radice al nodo che dovr\`a contenere il valore.
L'ufficio BD vi commissiona una modifica della struttura dati che permette
di tenere conto di quante volte lo stesso valore \`e stato inserito in
un nodo. Per questo motivo dovrete ampliare l'oggetto {\tt node} in modo
da poter memorizzare un campo aggiuntivo chiamato {\tt count} che
viene aggiornato nel modo seguente.
\begin{itemize}
\item quando {\tt node} \`e creato {\tt count} \`e posto uguale a $1$;
\item se viene eseguito l'inserimento di {\tt val} per un
nodo {\tt node} gi\`a esistente e {\tt val} \`e diverso dal valore corrente
di {\tt node.data}, allora
{\tt node.data} viene posto uguale a {\tt val} e {\tt node.count } 
\`e posto uguale a $1$.
\item se, invece, viene eseguito l'inserimento di {\tt val} per un
nodo {\tt node} gi\`a esistente e {\tt val} \`e uguale al valore corrente
di {\tt node.data}, allora
{\tt node.data} non viene modificato e  {\tt node.count } 
\`e incrementato.
\end{itemize}

Il compito consiste nel modificare la class {\tt tree}
in modo da supportare la gestione di {\tt count} come descritta sopra
e di modificare la visita {\tt inorder} in modo
da stampare anche il campo {\tt count} oltre al campo {\tt data}.


\medskip\noindent{\bf Materiale della traccia.}
La cartella contiene il pdf di questa traccia, il file
{\tt alberi.py} che contiene la classe sviluppata in classe,
il file {\tt driver.py} che pu\`o essere usato
per verificare il funzionamento della classe progettata, e il file  {\tt result.txt} che contiene
l'output atteso di {\tt driver.py}.

\medskip\noindent{\bf Istruzione per la consegna.}
Tutto il codice consegnato deve essere contenuto nel file
{\tt Sol.py} ed inviato per e-mail all'indirizzo
{\tt giuper@gmail.com} prima delle ore \oraconsegna\ di oggi, 
\dataoggi. Non inviare altri file e n\'e tantomeno file zip.
Il file deve contenere la classe {\tt tree}  
che pu\`o essere usata per eseguire
il codice di {\tt driver.py}. Si pu\`o assumere che il file
{\tt alberi.py} sia presente al momento dell'esecuzione. 




\end{document}
