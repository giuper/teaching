\documentclass{amsart}


\usepackage{fullpage}


\begin{document}
\title{Algoritmi e Strutture Dati\\
Statistica per i big data\\
Esame del 4 Gennaio 2021
}


\newcommand{\NomeStudente}{Giuseppe Persiano}
\newcommand{\nomeClasse}{{\tt{Sol}}}
\newcommand{\nomeMetodo}{{\tt{nuovo}}}
\newcommand{\oraconsegna}{10:35}
\newcommand{\dataoggi}{4 Gennaio, 2021}


\maketitle

\hfill{{\bf Studente: \NomeStudente}}

\smallskip
L'ufficio statistico del comune di Bugliano raccoglie i Big Data  della
ridente cittadina usando gli alberi binari di ricerca.
Per venire incontro ad una richiesta del capitano della locale squadra
di badminton, l'ufficio ci chiede di modificare la classe {\tt BST} 
studiata in classe nel modo seguente e di costruire una nuova classe
chiama {\nomeClasse}.
La procedura di inserimento {\tt insert} di un albero della classe 
{\nomeClasse} prende in input una stringa $S$.
La stringa verr\`a inserita in un nodo {\tt node} che ha 
{\tt node.key=len(S)} e ha {\tt node.val} uguale alla lista
di tutte le stringhe fin qui inserite nell'albero e che hanno
la stessa lunghezza di $S$.

In altre parole, per ogni nodo {\tt node} di un
albero della classe {\nomeClasse} abbiamo che
 {\tt node.key} \`e un intero
mentre {\tt node.val} \`e la lista di stringhe di lunghezza {\tt node.key}
che sono state inserite nell'albero.

\smallskip
Ad esempio, se le stringhe 
{\tt aaaa, abaa, aaca, aaad, abcd} sono inserite con successive operazioni 
di {\tt insert} in un albero inizialmente vuoto,
l'albero risultante avr\`a un solo nodo,
la radice, che ha {\tt key=4} (la lunghezza delle stringhe)
e {\tt val=['aaaa','abaa', 'aaca', 'aaad', 'abcd']}.
Se successivamente inseriamo la stringa {\tt bc}, 
la procedura {\tt insert } aggiunger\`a il figlio sinistro della radice
con {\tt key=2}  e {\tt val=['bc']}.


\medskip\noindent{\bf Materiale della traccia.}
La cartella contiene il pdf di questa traccia, il file
{\tt bst.py} che contiene la classe sviluppata in classe,
il file {\tt driver.py} che pu\`o essere usato
per verificare il funzionamento della classe progettata, e il file  {\tt result.txt} che contiene
l'output atteso di {\tt driver.py}.

\medskip\noindent{\bf Istruzione per la consegna.}
Tutto il codice consegnato deve essere contenuto nel file
{\tt \nomeClasse .py} ed inviato per e-mail all'indirizzo
{\tt giuper@gmail.com} prima delle ore \oraconsegna\ di oggi, 
\dataoggi. Non inviare altri file e n\'e tantomeno file zip.
Il file deve contenere la classe {\tt \nomeClasse}  
che pu\`o essere usata per eseguire
il codice di {\tt driver.py}. Si pu\`o assumere che il file
{\tt bst.py} sia presente al momento dell'esecuzione. 


\end{document}
