\documentclass{amsart}


\usepackage{fullpage}


\begin{document}
\title{Algoritmi e Strutture Dati\\
Statistica per i big data\\
Esame del 4 Gennaio 2021
}


\newcommand{\NomeStudente}{Giuseppe Persiano}
\newcommand{\nomeClasse}{{\tt{Sol}}}
\newcommand{\nomeMetodo}{{\tt{nuovo}}}
\newcommand{\oraconsegna}{10:35}
\newcommand{\dataoggi}{4 Gennaio, 2021}


\maketitle

\hfill{{\bf Studente: \NomeStudente}}

\smallskip
Il dottor Fabio Buggiani, sindaco del comune di Bugliano,
ha scoperto con rammarico che la visita inorder dell'implementazione
degli alberi binari di ricerca discussa in classe
stampa le chiavi mentre lui vorrebbe che fosse 
restituita una lista.

Vi chiede pertanto di aggiungere un nuovo metodo chiamato \nomeMetodo\
che prende in input un albero binario di ricerca e restituisce una lista
in cui tutte le chiavi compaiono nell'ordine in cui sono visitate
dalla visita {\tt preorder} (e s\`\i\ ha cambiato idea ed adesso \`e interessato alla {\tt preorder}).

\medskip\noindent{\bf Materiale della traccia.}
La cartella contiene il pdf di questa traccia, il file
{\tt bst.py} che contiene la classe sviluppata in classe,
il file {\tt driver.py} che pu\`o essere usato
per verificare il funzionamento della classe progettata, e il file  {\tt result.txt} che contiene
l'output atteso di {\tt driver.py}.

\medskip\noindent{\bf Istruzione per la consegna.}
Tutto il codice consegnato deve essere aggiunto al file
{\tt bst.py} ed inviato per e-mail all'indirizzo
{\tt giuper@gmail.com} prima delle ore \oraconsegna\ di oggi, 
\dataoggi. Non inviare altri file e n\'e tantomeno file zip.
La nuova versione del file {\tt bst.py} deve poter essere usata per eseguire
il codice di {\tt driver.py}. 


\end{document}
