\documentclass{amsart}


\usepackage{fullpage}


\begin{document}
\title{Algoritmi e Strutture Dati\\
Statistica per i big data\\
Esame del 4 Gennaio 2021
}


\newcommand{\NomeStudente}{Giuseppe Persiano}
\newcommand{\nomeClasse}{{\tt{Sol}}}
\newcommand{\nomeMetodo}{{\tt{nuovo}}}
\newcommand{\oraconsegna}{10:35}
\newcommand{\dataoggi}{4 Gennaio, 2021}


\maketitle

\hfill{{\bf Studente: \NomeStudente}}

\smallskip
L'ufficio statistico del comune di Bugliano raccoglie i dati della
recente epidemia usando una lista con puntatori implementata mediante la 
classe {\tt{LinkedList}} discussa durante il corso.
Nello specifico, l'ufficio ha una lista in cui ogni nodo contiene una
lista di due elementi: il primo \`e la data mentre il secondo \`e il
numero di guariti relativo alla data.
La lista \`e ordinata per data crescente. 

Il compito consiste nel costruire una nuova classe \nomeClasse\ che 
aggiunge a {\tt{LinkedList}} il metodo \nomeMetodo\
che costruisce la lista {\em cumulata} cos\`\i\ definita.
Data una lista di $n$ nodi $L$, la lista cumulata $S$ consiste di $n$
nodi. L'$i$-esimo nodo della lista cumulata $S$ contiene una lista
di due elementi:
\begin{itemize}
\item la data dell'$(i)$-esimo nodo di $L$;
\item la somma di tutti i guariti dall'inizio della lista $L$ fino al
nodo $(i)$-esimo di $L$ incluso.
\end{itemize}
In altre parole, la lista cumulata riporta, per ogni giorno,
il numero totale di guariti fino alla data.
Ad esempio, 
se la lista $L$ contiene i dati nell'ordine seguente
\begin{enumerate}
\item '1 Dicembre 2020', 2
\item '2 Dicembre 2020', 3
\item '3 Dicembre 2020', 4
\item '4 Dicembre 2020', 7
\item '5 Dicembre 2020', 2
\item '6 Dicembre 2020', 21
\item '7 Dicembre 2020', 12
\end{enumerate}
la lista cumulata $S$ contiene i seguenti dati
\begin{enumerate}
\item '1 Dicembre 2020', 2
\item '2 Dicembre 2020', 5
\item '3 Dicembre 2020', 9
\item '4 Dicembre 2020', 16
\item '5 Dicembre 2020', 18
\item '6 Dicembre 2020', 39
\item '7 Dicembre 2020', 51
\end{enumerate}


\medskip\noindent{\bf Materiale della traccia.}
La cartella contiene il pdf di questa traccia, il file
{\tt linkedList.py} che contiene la classe sviluppata in classe,
il file {\tt driver.py} che pu\`o essere usato
per verificare il funzionamento della classe progettata, e il file  {\tt result.txt} che contiene
l'output atteso di {\tt driver.py}.

\medskip\noindent{\bf Istruzione per la consegna.}
Tutto il codice consegnato deve essere contenuto nel file
{\tt \nomeClasse .py} ed inviato per e-mail all'indirizzo
{\tt giuper@gmail.com} prima delle ore \oraconsegna\ di oggi, 
\dataoggi. Non inviare altri file e n\'e tantomeno file zip.
Il file deve contenere la classe {\tt \nomeClasse}  
che offre il metodo $\nomeMetodo$ e che pu\`o essere usata per eseguire
il codice di {\tt driver.py}. Si pu\`o assumere che il file
{\tt linkedList.py} sia presente al momento dell'esecuzione. 



\end{document}
