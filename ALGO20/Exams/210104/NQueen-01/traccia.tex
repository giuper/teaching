\documentclass{amsart}


\usepackage{fullpage}


\begin{document}
\title{Algoritmi e Strutture Dati\\
Statistica per i big data\\
Esame del 4 Gennaio 2021
}


\newcommand{\NomeStudente}{Giuseppe Persiano}
\newcommand{\nomeClasse}{{\tt{Sol}}}
\newcommand{\nomeMetodo}{{\tt{nuovo}}}
\newcommand{\oraconsegna}{10:35}
\newcommand{\dataoggi}{4 Gennaio, 2021}


\maketitle

\hfill{{\bf Studente: \NomeStudente}}

\smallskip
Vittorino, il capo dei vigili urbani del comune di Bugliano, passa il tempo,
tra una multa e l'altra, a giocare a scacchi. Ha trovato sul sito del corso
il codice per il problema delle $n$ regine e l'ha trovato interessante.
In questo periodo di pandemia per\`o \`e diventato molto sensibile al 
distanziamento sociale e quindi vi ha chiesto di creare
una nuova classe {\nomeClasse} che calcola, per un dato numero 
$n$ di regine, se esiste un modo di piazzare $n$ regine  su una scacchiera 
di lato $n$ in modo che non si attacchino (e fin qui \`e simile al problema 
originario) ed in pi\`u se due regine sono su righe consecutive allora si trovano
ad almeno $3$ colonne di distanza.

\medskip\noindent{\bf Materiale della traccia.}
La cartella contiene il pdf di questa traccia, i file
{\tt stack.py, back.py, queen.py},
il file {\tt driver.py} che pu\`o essere usato
per verificare il funzionamento della classe progettata
e il file  {\tt result.txt} che contiene l'output atteso di {\tt driver.py}.

\medskip\noindent{\bf Istruzione per la consegna.}
Tutto il codice consegnato deve essere contenuto nel file
{\tt \nomeClasse .py} ed inviato per e-mail all'indirizzo
{\tt giuper@gmail.com} prima delle ore \oraconsegna\ di oggi, 
\dataoggi. Non inviare altri file e n\'e tantomeno file zip.
Il file deve contenere la classe {\tt \nomeClasse}  
che pu\`o essere usata per eseguire
il codice di {\tt driver.py}. Si pu\`o assumere che i file
{\tt stack.py, back.py, queen.py}  
siano presenti al momento dell'esecuzione. 


\end{document}
