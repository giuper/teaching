\documentclass{amsart}


\begin{document}
\title{Algoritmi e Strutture Dati\\
Statistica per i big data\\
Esame del 16 Luglio 2020
}


\newcommand{\NomeStudente}{sol}
\newcommand{\nomeClasse}{sol}


\maketitle

\hfill{{\bf Studente: Giuseppe Persiano}}

\smallskip

In classe abbiamo discusso una semplice classe python che risolve il 
problema delle $N$ regine.
In questo problema viene chiesto di risolvere una generalizzazione 
del problema  delle $N$ regine con l'ulteriore vincolo che ci sono 
delle posizioni vietate sulla scacchiera in cui non \`e possibile 
posizionare una regina.


\medskip\noindent{\bf Istruzione per la consegna.}
Tutto il codice consegnato deve essere contenuto nel file
{\tt \nomeClasse .py} ed inviato per e-mail all'indirizzo
{\tt giuper@gmail.com} prima delle ore 11 di oggi, 16 Luglio 2020.

Il file deve contenere una classe {\tt \nomeClasse} che deve offrire
i seguenti due metodi:
\begin{enumerate}
\item {\tt \_\_init\_\_} che prende in input la grandezza $n$ della scacchiera
e la lista delle posizioni proibite. Ogni posizione proibita consiste di una lista di due interi corrispondenti alla riga e alla colonna della posizione
proibita.
\item {\tt solve} che risolve un'istanza del problema.
\end{enumerate}

La cartella contiene il pdf di questa traccia, il file
{\tt nRegine.py} che contiene la classe sviluppata in classe,
il file {\tt driver.py} che pu\`o essere usato
per verificare il funzionamento della classe progettata, e il file  {\tt result.txt} che contiene
un possibile output di {\tt driver.py}.



\end{document}
